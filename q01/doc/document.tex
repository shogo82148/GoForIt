\documentclass{jsarticle}

\title{人生の時計}
\author{Ichinose Shogo}

\begin{document}
\maketitle

\section{アルゴリズムの説明}
一生を一日に例えた場合の午前0時からの経過時間を以下の式を使って求めます。
\[
\textrm{1日に例えた場合の0時からの経過時間[日]} = \frac{\textrm{現在時刻}-\textrm{誕生日}}{\textrm{亡くなる時刻}-\textrm{誕生日}}
\]

日時の計算および、経過時間を何時何分何秒の形式に変換するのには、Pythonの標準モジュールであるdatetimeを使用しました。

\section{動作方法}
実行にはPythonが必要です。追加モジュールは必要ありません。

実行すると生まれた年、月、日、生きることのできる年齢を標準入力から入力すると、一生を24時間に例えた時の時刻が表示されます。

\section{出力例}
以下に出力例を示します。最後の行がプログラムからの出力で、他は入力です。
\begin{verbatim}
    $ python q01.py
    Birth Year?:1988
    Birth Month?:10
    Birth Day?:17
    How many years will you live?:100
     5:35:38.813720
\end{verbatim}

\section{ライセンス}
著作権はIchinose Shogoが保有します。
MITライセンスにしたがって配布します。

\end{document}
