\documentclass{jsarticle}

\title{旋律に隠された特徴}
\author{Ichinose Shogo}

\begin{document}
\maketitle

\section{アルゴリズムの説明}
旋律中のすべての音符に対し、
「ABS(今着目している音符の音の高さ-一つ前の音符の音の高さ)+今着目している音符の長さ/一つ前の音符の長さ-1」
を計算し、総和を求めます。その計算結果の整数部を特徴量としました。

\section{動作方法}
動作にはPython2.xが必要です。

\section{出力例}
以下に入出力例を示します。
\begin{verbatim}
$ python q04.py
i)
A 20
B 22
C 20

ii)
D 22
E 28
F 21
D has same feature to B

iii)
G 57
result:
-2:1:,14:1:,-2:0.:
Feature 57
\end{verbatim}

\section{ライセンス}
著作権はIchinose Shogoが保有します。
MITライセンスにしたがって配布します。

\end{document}
