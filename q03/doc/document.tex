\documentclass{jsarticle}

\title{暗号検索の高速化}
\author{Ichinose Shogo}

\begin{document}
\maketitle

\section{アルゴリズムの説明}

\begin{enumerate}
\item キーワードの先頭と末尾の文字をランダム文字列から検索し、位置を記憶します。\\
	$w$をキーワード、$r$をランダム文字列、$n=|w|$とした時、先頭位置$h$と末尾位置$t$を計算します。
	\begin{eqnarray*}
		H &=& \{ i | r_i = w_0 \} \\
		T &=& \{ i | r_i = w_{n-1} \}
	\end{eqnarray*}
\item 全ての先頭位置と末尾位置の組み合わせ($(t,h) \in H \times T$)について、正解かどうかを確かめます。
	$s = (t-h) / (n-1)$ とした時、すべての$i(0 \le i < n)$ について $r_{h+si} = w_i$ であれば、正解一覧に追加します。
\item 正解だとわかった組み合わせを、スキップ数にソートし、出力します。
\end{enumerate}
ステップ2で組み合わせを作る際、位置 mod (キーワードの長さ-1) が等しいもの同士のみで組み合わせる事によって、
候補数を削減します。

\section{動作方法}
キーワードを引数として渡すと、
標準入力から入力された文字列から、キーワードを探して、
見つかった位置を標準出力に出力します。

\section{出力例}
以下に入出力例を示します。
\begin{verbatim}
$ g++ -o q03 q03.cpp
$ python random_string.py > random.txt
$ ./q03 sony < random.txt > result.txt
1,118641
2,146003
2,235908
2,284065
3,288544
...
\end{verbatim}

\section{ライセンス}
著作権はIchinose Shogoが保有します。
MITライセンスにしたがって配布します。

\end{document}
